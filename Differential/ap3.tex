\documentclass{article}
\usepackage[utf8]{inputenc}
\usepackage{cancel}
\usepackage{amsmath}
\usepackage{pgfplots}
\usepackage{amsfonts}

\title{AP3 de Cálculo I}
\author{Vinícius Menezes Monte}
\date{Maio de 2022}

\begin{document}

\maketitle

\section*{Questões}
\subsection*{Calcule a derivada indicada: (4 escores cada)}

\subsubsection*{1.}

\[
    f(x) = x^2 + x \cdot \cos x + \pi , \quad f'(x)
\]

\subsubsection*{2.}

\[
    g(x) = \ln x - \frac{3^x}{\sin x} - \frac{1}{x}, \quad g'(x)
\]

\subsubsection*{3.}

\[
    \phi(\theta) = \cosh(\sin e^{\theta}) - \sinh(\cos e^{\theta}), \quad \phi'(\theta)
\]

\subsubsection*{4.}

\[
    y = x^{72} + \cos x + e^x, \quad \frac{d^{72}y}{dx^{72}}
\]

\subsubsection*{5.}

\[
    y - x^2y^2 - \cos(xy) = 4, \quad \frac{dy}{dx} \quad e \quad \frac{dx}{dy}
\]

\subsubsection*{6. Encontre uma equação da reta tangente e uma equação da reta normal à curva \(y = \sqrt{x} + \frac{1}{x^2}\) no ponto de abscissa \(x=1\).}

\section*{Soluções}

\subsubsection*{1.}

\[
    f(x) = x^2 + x \cdot \cos x + \pi , \quad f'(x)
\]

A derivada da soma é a soma das derivadas:

\begin{equation}\label{eq:q1_soma}
    f'(x) = \frac{d (x^2)}{dx} + \frac{d (x\cos x)}{dx} + \frac{d(\pi)}{dx}
\end{equation}

Portanto, voltemos nossa atenção para cada termo da soma.

A derivada de uma constante é zero. Como \(\pi\) é uma constante, a derivada de \(\pi\) é zero.

\begin{equation}\label{eq:derivative_pi_is_zero}
    \frac{d(\pi)}{dx} = 0
\end{equation}

A derivada da função potencial \(p(x) = x^n\) é  \(p'(x) = nx^{n-1}\). Desse modo

\begin{equation}\label{eq:derivative_x_to_the_2}
    \frac{d(x^2)}{dx} = 2x
\end{equation}

Para diferenciar o termo restante, utilizaremos a Regra do Produto.

A Regra do Produto diz que se

\[
    f(x) = g(x)\cdot h(x)
\]

então

\[
    f'(x) = g'(x)\cdot h(x) + g(x)\cdot h'(x)
\]

Ou ainda, usando a notação de Leibniz:

\[
    y = f(x)
\]

\[
    u = g(x)
\]

\[
    v = h(x)
\]

\[
    y = u\cdot v \implies \frac{dy}{dx} = \frac{du}{dx} \cdot v + u \cdot \frac{dv}{dx}
\]

Com isso em vista, analisemos a expressão

\[
    \frac{d (x\cos x)}{dx}
\]

\begin{align*}
    \frac{d (x\cos x)}{dx} = \frac{dx}{dx} \cdot \cos x + x \cdot \frac{d (\cos x)}{dx}
\end{align*}

Pelas propriedades da função potencial

\[
    \frac{dx}{dx} = \frac{d(x^1)}{dx} = 1 \cdot x^{1-1} = 1 \cdot x^0 = 1 \cdot 1 = 1
\]

A derivada do cosseno é menos seno.

\[
    \frac{d (\cos x)}{dx} = - \sin x
\]

Portanto

\begin{align*}
    \frac{d (x\cos x)}{dx}
     & = 1 \cdot \cos x + x \cdot - \sin x \\
     & = \cos x - x \cdot \sin x
\end{align*}

\begin{equation}\label{eq:q1_product_rule}
    \frac{d (x\cos x)}{dx} = \cos x - x \cdot \sin x
\end{equation}

De acordo com as expressões
\ref{eq:q1_soma},
\ref{eq:derivative_pi_is_zero},
\ref{eq:derivative_x_to_the_2}
e \ref{eq:q1_product_rule} temos que

\begin{align*}
    f'(x)
     & = \frac{d (x^2)}{dx} + \frac{d (x\cos x)}{dx} + \frac{d(\pi)}{dx} \\
     & = 2x + \cos x - x \cdot \sin x + 0                                \\
     & = 2x + \cos x - x \cdot \sin x                                    \\
     & = \cos x - x (\sin x - 2)
\end{align*}

Finalmente:

\[
    f'(x) = \cos x - x (\sin x - 2)
\]


\subsubsection*{2.}

\[
    g(x) = \ln x - \frac{3^x}{\sin x} - \frac{1}{x}, \quad g'(x)
\]

Como a derivada da soma é a soma das derivadas, diferenciemos a soma termo a termo e fatoremos as constantes.

\begin{equation}\label{eq:q2_diferenca}
    g'(x) = \frac{d (\ln x)}{dx}
    - \frac{d (\frac{3^x}{\sin x})}{dx}
    - \frac{d(\frac{1}{x})}{dx}
\end{equation}

Portanto, voltemos nossa atenção para cada termo da diferença.

É imediato que

\begin{equation}\label{eq:q2_derivada_lnx}
    \frac{d (\ln x)}{dx} = \frac{1}{x}
\end{equation}

Vamos reescrever a expressão \(\frac{3^x}{\sin x}\)

\begin{align*}
    \frac{3^x}{\sin x} = 3^x \cdot \frac{1}{\sin x} = 3^x \cdot \csc x
\end{align*}

Desse modo, temos que

\[
    \frac{d (\frac{3^x}{\sin x})}{dx} = \frac{d (3^x \cdot \csc x)}{dx}
\]

Apliquemos a Regra do Produto, descrita anteriormente:

\begin{align*}
    \frac{d (\frac{3^x}{\sin x})}{dx}
     & = \frac{d (3^x \cdot \csc x)}{dx}                                    \\
     & = \frac{d (3^x)}{dx} \cdot \csc x +  3^x \cdot \frac{d (\csc x)}{dx}
\end{align*}

Para continuarmos a aplicação da Regra do Produto,
nos atentemos a algumas derivadas notáveis.

Seja a função exponencial \(f_e(x) = a^x\).
Sua derivada é \(f_e'(x) = a^x \cdot \ln a\)

Seja a função cossecante \(f_{c}(x) = \csc x\).
Sua derivada é \(f_{c}'(x) = -\csc x \cdot \cot x\)

Agora, continuemos a aplicação da Regra do Produto.

\begin{align*}
    \frac{d (\frac{3^x}{\sin x})}{dx}
     & = \frac{d (3^x)}{dx} \cdot \csc x +  3^x \cdot \frac{d (\csc x)}{dx} \\
     & = (3^x \cdot \ln 3) \cdot \csc x + 3^x \cdot (-\csc x \cdot \cot x)  \\
     & = 3^x \cdot \ln 3 \cdot \csc x - 3^x \cdot \csc x \cdot \cot x       \\
     & = 3^x \cdot \csc x  \cdot \ln 3 - 3^x \cdot \csc x \cdot \cot x      \\
     & = (3^x \cdot \csc x)  \cdot \ln 3 - (3^x \cdot \csc x) \cdot \cot x  \\
     & = (3^x \cdot \csc x) \cdot (\ln 3 - \cot x)                          \\
\end{align*}

Ou seja

\begin{equation}\label{eq:q2_derivada_3_to_the_x_over_sin_x}
    \frac{d (\frac{3^x}{\sin x})}{dx} = (3^x \cdot \csc x) \cdot (\ln 3 - \cot x)
\end{equation}

Para diferenciar o termo restante, utilizaremos as propriedades da função potencial.

\begin{align*}
    \frac{d(\frac{1}{x})}{dx}
    = \frac{d(x^{-1})}{dx}
    = -1 \cdot x^{-1 - 1}
    = -x^{-2}
    = -\frac{1}{x^2}
\end{align*}

Ou seja

\begin{equation}\label{eq:q2_derivada_1_over_x}
    \frac{d (\frac{1}{x})}{dx} = -\frac{1}{x^2}
\end{equation}

Das expressões
\ref{eq:q2_diferenca},
\ref{eq:q2_derivada_lnx},
\ref{eq:q2_derivada_3_to_the_x_over_sin_x} e
\ref{eq:q2_derivada_1_over_x}
decorre que

\begin{align*}
    g'(x)
     & =
    \frac{d (\ln x)}{dx}
    - \frac{d (\frac{3^x}{\sin x})}{dx}
    - \frac{d(\frac{1}{x})}{dx}
    \\
     & =
    \left(\frac{1}{x}\right)
    - ((3^x \cdot \csc x) \cdot (\ln 3 - \cot x))
    - \left(-\frac{1}{x^2}\right)
    \\
     & =
    \frac{1}{x}
    + \frac{1}{x^2}
    - ((3^x \cdot \csc x) \cdot (\ln 3 - \cot x))
    \\
     & =
    \frac{x + 1}{x^2}
    - (3^x \cdot \csc x  \cdot \ln 3
    - 3^x \cdot \csc x \cdot \cot x)
    \\
     & =
    \frac{x + 1}{x^2}
    + 3^x \cdot \csc x \cdot \cot x
    - 3^x \cdot \csc x  \cdot \ln 3
\end{align*}

Finalmente, temos que:

\[
    g'(x) =
    \frac{x + 1}{x^2}
    + 3^x \cdot \csc x \cdot \cot x - 3^x \cdot \csc x  \cdot \ln 3
\]


\subsubsection*{3.}

\[
    \phi(\theta) = \cosh(\sin e^{\theta}) - \sinh(\cos e^{\theta}), \quad \phi'(\theta)
\]

Como a derivada da soma é a soma das derivadas, diferenciemos a soma termo a termo e fatoremos as constantes.

\begin{equation}\label{eq:q3_diferenca}
    \phi'(\theta) = \frac{d(\cosh(\sin e^{\theta}))}{d\theta}
    - \frac{d(\sinh(\cos e^{\theta}))}{d\theta}
\end{equation}

Portanto, voltemos nossa atenção para cada termo da diferença.

Analisemos o primeiro termo.

\[
    \frac{d(\cosh(\sin e^{\theta}))}{d\theta}
\]

Vamos aplicar a Regra da Cadeia. Primeiro, analisemos o número
de funções que figuram no primeiro termo.

Encontramos 3 funções:

\begin{itemize}
    \item A função cosseno hiperbólico: \(\cosh\)
    \item A função seno: \(\sin\)
    \item A função exponencial de base e: \(e^{\theta}\)
\end{itemize}

De acordo com essa análise, sabemos que usaremos a
Regra da Cadeia na sua forma para três funções, que é a seguinte:

\[
    \frac{dy_1}{d \theta}
    =
    \frac{dy_1}{du_1} \cdot \frac{du_1}{dv_1} \cdot \frac{dv_1}{d \theta}
\]

Façamos

\begin{itemize}
    \item \(y_1(u_1) = \cosh u_1\)
    \item \(u_1(v_1) = \sin v_1 \)
    \item \(v_1(\theta) = e^{\theta}\)
\end{itemize}

Desse modo

\begin{align*}
    \frac{dy_1}{d \theta}
     & = \frac{d(\cosh u_1)}{du_1} \cdot \frac{d(\sin v_1)}{dv_1} \cdot \frac{d(e^{\theta})}{d \theta}
\end{align*}

Para continuarmos a aplicação da Regra da Cadeia,
nos atentemos a algumas derivadas notáveis.

\begin{itemize}
    \item A derivada da função cosseno hiperbólico é a função seno hiperbólico:
          \[
              (\cosh x)' = \sinh x
          \]
    \item A derivada da função seno é a função cosseno:
          \[
              (\sin x)' = \cos x
          \]
    \item A derivada da função exponencial de base e é ela própria:
          \[
              (e^x)' = e^x
          \]
\end{itemize}

Agora, continuemos a aplicação da Regra da Cadeia.

\begin{align*}
    \frac{dy_1}{d \theta}
     & =
    \frac{d(\cosh u_1)}{du_1} \cdot
    \frac{d(\sin v_1)}{dv_1} \cdot
    \frac{d(e^{\theta})}{d \theta}
    \\ & =
    (\sinh u_1) \cdot
    (\cos v_1) \cdot
    (e^{\theta})
    \\ & =
    \sinh (\sin v_1) \cdot
    \cos v_1 \cdot
    e^{\theta}
    \\ & =
    \sinh (\sin e^{\theta}) \cdot
    \cos e^{\theta} \cdot
    e^{\theta}
\end{align*}

Ou seja:

\begin{equation}\label{eq:q3_derivada_1}
    \frac{d(\cosh(\sin e^{\theta}))}{d\theta}
    =
    \sinh (\sin e^{\theta}) \cdot
    \cos e^{\theta} \cdot
    e^{\theta}
\end{equation}

Para o segundo termo, aplicamos procedimento análogo.

Analisemos o segundo termo.

\[
    \frac{d(\sinh(\cos e^{\theta}))}{d\theta}
\]

Vamos aplicar a Regra da Cadeia. Primeiro, analisemos o número
de funções que figuram no segundo termo.

Encontramos 3 funções:

\begin{itemize}
    \item A função seno hiperbólico: \(\sinh\)
    \item A função cosseno: \(\cos\)
    \item A função exponencial de base e: \(e^{\theta}\)
\end{itemize}

De acordo com essa análise, sabemos que usaremos a
Regra da Cadeia na sua forma para três funções, que é a seguinte:

\[
    \frac{dy_2}{d \theta}
    =
    \frac{dy_2}{du_2} \cdot \frac{du_2}{dv_2} \cdot \frac{dv_2}{d \theta}
\]

Façamos

\begin{itemize}
    \item \(y_2(u_2) = \sinh u_2\)
    \item \(u_2(v_2) = \cos v_2 \)
    \item \(v_2(\theta) = e^{\theta}\)
\end{itemize}

Desse modo

\begin{align*}
    \frac{dy_2}{d \theta}
     & = \frac{d(\sinh u_2)}{du_2} \cdot \frac{d(\cos v_2)}{dv_2} \cdot \frac{d(e^{\theta})}{d \theta}
\end{align*}

Para continuarmos a aplicação da Regra da Cadeia,
nos atentemos a algumas derivadas notáveis.

\begin{itemize}
    \item A derivada da função seno hiperbólico é a função cosseno hiperbólico:
          \[
              (\sinh x)' = \cosh x
          \]
    \item A derivada da função cosseno é a função menos seno:
          \[
              (\cos x)' = -\sin x
          \]
    \item Como já visto, a derivada da função exponencial de base e é ela própria:
          \[
              (e^x)' = e^x
          \]
\end{itemize}

Agora, continuemos a aplicação da Regra da Cadeia.

\begin{align*}
    \frac{dy_2}{d \theta}
     & =
    \frac{d(\sinh u_2)}{du_2} \cdot
    \frac{d(\cos v_2)}{dv_2} \cdot
    \frac{d(e^{\theta})}{d \theta}
    \\ & =
    (\cosh u_2) \cdot
    (-\sin v_2) \cdot
    (e^{\theta})
    \\ & =
    -
    \cosh (\cos v_2) \cdot
    \sin v_2 \cdot
    e^{\theta}
    \\ & =
    -
    \cosh (\cos e^{\theta}) \cdot
    \sin e^{\theta} \cdot
    e^{\theta}
\end{align*}

Ou seja:

\begin{equation}\label{eq:q3_derivada_2}
    \frac{d(\sinh(\cos e^{\theta}))}{d\theta}
    =
    -
    \cosh (\cos e^{\theta}) \cdot
    \sin e^{\theta} \cdot
    e^{\theta}
\end{equation}

Das expressões
\ref{eq:q3_diferenca},
\ref{eq:q3_derivada_1} e
\ref{eq:q3_derivada_2}
decorre que

\begin{align*}
    \phi'(\theta)
     & =
    \frac{d(\cosh(\sin e^{\theta}))}{d\theta}
    - \frac{d(\sinh(\cos e^{\theta}))}{d\theta}
    \\ &=
    (
    \sinh (\sin e^{\theta}) \cdot
    \cos e^{\theta} \cdot
    e^{\theta}
    )
    -
    (
    -
    \cosh (\cos e^{\theta}) \cdot
    \sin e^{\theta} \cdot
    e^{\theta}
    )
    \\ &=
    \sinh (\sin e^{\theta}) \cdot
    \cos e^{\theta} \cdot
    e^{\theta}
    +
    \cosh (\cos e^{\theta}) \cdot
    \sin e^{\theta} \cdot
    e^{\theta}
    \\ &=
    e^{\theta}
    \cdot
    (
    \sinh (\sin e^{\theta}) \cdot
    \cos e^{\theta}
    +
    \cosh (\cos e^{\theta}) \cdot
    \sin e^{\theta}
    )
\end{align*}

Finalmente:

\[
    \phi'(\theta)
    =
    e^{\theta}
    \cdot
    (
    \sinh (\sin e^{\theta}) \cdot
    \cos e^{\theta}
    +
    \cosh (\cos e^{\theta}) \cdot
    \sin e^{\theta}
    )
\]

\subsubsection*{4.}

\[
    y = x^{72} + \cos x + e^x, \quad \frac{d^{72}y}{dx^{72}}
\]

A derivada da soma é a soma das derivadas.

\begin{equation}\label{eq:q4_soma}
    \frac{d^{72}y}{dx^{72}}
    =
    \frac{d^{72}(x^{72})}{dx^{72}}
    +
    \frac{d^{72}(\cos x )}{dx^{72}}
    +
    \frac{d^{72}(e^x)}{dx^{72}}
\end{equation}

Portanto, voltemos nossa atenção para cada termo da soma.

Analisemos o termo \(\frac{d^{72}(x^{72})}{dx^{72}} \).

Considere as seguintes derivadas:

\begin{align*}
    \frac{d(x^{72})}{dx}           & = 72x^{71}
    \\
    \frac{d^2(x^{72})}{dx^2}       & = 72 \cdot 71 \cdot x^{70}
    \\
    \frac{d^3(x^{72})}{dx^3}       & = 72 \cdot 71 \cdot 70 \cdot x^{69}
    \\
    \frac{d^4(x^{72})}{dx^4}       & = 72 \cdot 71 \cdot 70 \cdot 69 \cdot x^{68}
    \\
    \frac{d^5(x^{72})}{dx^5}       & = 72 \cdot 71 \cdot 70 \cdot 69 \cdot 68 \cdot x^{67}
    \\
                                   & \dots
    \\
    \frac{d^n(x^{72})}{dx^n}       & = \frac{72!}{(72 - n)!} \cdot x^{72 - n}
    \\
                                   & \dots
    \\
    \frac{d^{72}(x^{72})}{dx^{72}} & = \frac{72!}{(72 - 72)!} \cdot x^{72 - 72}
    \\
    \frac{d^{72}(x^{72})}{dx^{72}} & = \frac{72!}{(0)!} \cdot x^{0}
    \\
    \frac{d^{72}(x^{72})}{dx^{72}} & = \frac{72!}{1} \cdot 1
    \\
    \frac{d^{72}(x^{72})}{dx^{72}} & = 72!
\end{align*}

Portanto:

\begin{equation}\label{eq:q4_derivada_1}
    \frac{d^{72}(x^{72})}{dx^{72}} = 72!
\end{equation}

Analisemos o termo \( \frac{d^{72}(\cos x )}{dx^{72}}  \).

Considere as seguintes derivadas:

\begin{align*}
    \frac{d(\cos x )}{dx}           & = - \sin x
    \\
    \frac{d^2(\cos x )}{dx^2}       & = - \cos x
    \\
    \frac{d^3(\cos x )}{dx^3}       & = - ( - \sin x) = \sin x
    \\
    \frac{d^4(\cos x )}{dx^4}       & = \cos x, \quad 4 = 4 \cdot 1
    \\
    \frac{d^5(\cos x )}{dx^5}       & = - \sin x
    \\
    \frac{d^6(\cos x )}{dx^6}       & = - \cos x
    \\
    \frac{d^7(\cos x )}{dx^7}       & = \sin x
    \\
    \frac{d^8(\cos x )}{dx^8}       & = \cos x, \quad 8 = 4 \cdot 2
    \\
    \frac{d^9(\cos x )}{dx^9}       & = - \sin x
    \\
    \frac{d^{10}(\cos x )}{dx^{10}} & = - \cos x
    \\
    \frac{d^{11}(\cos x )}{dx^{11}} & = \sin x
    \\
    \frac{d^{12}(\cos x )}{dx^{12}} & = \cos x, \quad 12 = 4 \cdot 3
    \\
                                    & \dots
    \\
    \frac{d^{4k}(\cos x )}{dx^{4k}} & = \cos x, \quad 4k = 4 \cdot k
    \\
                                    & \dots
    \\
    \frac{d^{72}(\cos x )}{dx^{72}} & = \cos x, \quad 72 = 4 \cdot 16
    \\
                                    & \dots
\end{align*}

Perceba que a derivada de 4k-ésima ordem,
\(k \in \mathbb{N} \), da função \( \cos x \) é igual a \( \cos x \).
Como \(72 = 4k\) com \(k = 16\) e \(16 \in \mathbb{N}\), temos que:

\begin{equation}\label{eq:q4_derivada_2}
    \frac{d^{72}(\cos x )}{dx^{72}} = \cos x
\end{equation}

Analisemos o termo \( \frac{d^{72}(e^x)}{dx^{72}} \).

Como a derivada da função exponencial de base
\(e\) é a própria função exponencial de base \(e\),
ao calcular as derivadas de ordens superiores da função
exponencial de base \(e\), sempre obteremos ela própria.
Desse modo

\begin{equation}\label{eq:q4_derivada_3}
    \frac{d^{72}(e^x)}{dx^{72}} = e^x
\end{equation}

Das expressões
\ref{eq:q4_soma},
\ref{eq:q4_derivada_1},
\ref{eq:q4_derivada_2} e
\ref{eq:q4_derivada_3}
decorre que

\begin{align*}
    \frac{d^{72}y}{dx^{72}}
     & =
    \frac{d^{72}(x^{72})}{dx^{72}}
    +
    \frac{d^{72}(\cos x )}{dx^{72}}
    +
    \frac{d^{72}(e^x)}{dx^{72}}
    \\ &=
    (72!)
    +
    (\cos x)
    +
    (e^x)
    \\ &=
    72!
    +
    \cos x
    +
    e^x
\end{align*}

Finalmente:

\[
    \frac{d^{72}y}{dx^{72}} = \cos x + e^x + 72!
\]

\subsubsection*{5.}

\[
    y - x^2y^2 - \cos xy = 4, \quad \frac{dy}{dx} \quad e \quad \frac{dx}{dy}
\]

Apliquemos a derivação implícita.

\[
    \frac{d(y - x^2y^2 - \cos xy)}{dx} = \frac{d(4)}{dx}
\]

Diferenciemos a soma termo a termo e fatoremos as constantes.

\[
    - \frac{d(\cos xy)}{dx} +  \frac{dy}{dx} - \frac{d(x^2y^2)}{dx}  = \frac{d(4)}{dx}
\]

Apliquemos a Regra da Cadeia

\[
    \frac{d(\cos xy)}{dx}
    =
    \frac{d(\cos u)}{du}
    \cdot
    \frac{du}{dx}
    ,\quad
    \text{onde}
    \quad
    u = xy
    \quad
    \text{e}
    \quad
    \frac{d(\cos u)}{du} = -\sin u
    \quad
    \text{:}
\]

\[
    \frac{dy}{dx} - \frac{d(x^2y^2)}{dx} - \left(- \frac{d(xy)}{dx} \cdot  \sin xy \right)  = \frac{d(4)}{dx}
\]

Simplifiquemos a expressão:

\[
    \frac{dy}{dx} - \frac{d(x^2y^2)}{dx} + \frac{d(xy)}{dx} \cdot  \sin xy   = \frac{d(4)}{dx}
\]

Apliquemos a Regra do Produto

\[
    \frac{d(uv)}{dx}
    =
    v \cdot \frac{du}{dx}
    +
    u \cdot \frac{dv}{dx},
    \quad
    \text{onde}
    \quad
    u = x
    \quad
    \text{e}
    \quad
    v = y
    \quad
    \text{:}
\]

\[
    \frac{dy}{dx}
    -
    \frac{d(x^2y^2)}{dx}
    +
    \left(
    y
    \cdot
    \frac{dx}{dx}
    +
    x
    \cdot
    \frac{dy}{dx}
    \right)
    \cdot
    \sin xy
    =
    \frac{d(4)}{dx}
\]

Organizemos a expressão

\[
    -
    \frac{d(x^2y^2)}{dx}
    +
    \sin xy
    \cdot
    \left(
    x
    \cdot
    \frac{dy}{dx}
    +
    \frac{dx}{dx}
    \cdot
    y
    \right)
    +
    \frac{dy}{dx}
    =
    \frac{d(4)}{dx}
\]

Apliquemos a Regra do Produto

\[
    \frac{d(uv)}{dx}
    =
    v
    \cdot
    \frac{du}{dx}
    +
    u
    \cdot
    \frac{dv}{dx}
    ,
    \quad
    \text{onde}
    \quad
    u = x^2
    \quad
    \text{e}
    \quad
    v = y^2
    \quad
    \text{:}
\]

\[
    -
    \left(
    x^2
    \cdot
    \frac{d(y^2)}{dx}
    +
    y^2
    \cdot
    \frac{d(x^2)}{dx}
    \right)
    +
    \sin xy
    \cdot
    \left(
    x
    \cdot
    \frac{dy}{dx}
    +
    \frac{dx}{dx}
    \cdot
    y
    \right)
    +
    \frac{dy}{dx}
    =
    \frac{d(4)}{dx}
\]

Simplifiquemos a expressão:

\[
    -
    x^2
    \cdot
    \frac{d(y^2)}{dx}
    -
    \frac{d(x^2)}{dx}
    \cdot
    y^2
    +
    \sin xy
    \cdot
    \left(
    x
    \cdot
    \frac{dy}{dx}
    +
    \frac{dx}{dx}
    \cdot
    y
    \right)
    +
    \frac{dy}{dx}
    =
    \frac{d(4)}{dx}
\]

Apliquemos a Regra da Cadeia

\[
    \frac{d y^2}{dx}
    =
    \frac{d(u^2)}{du}
    \cdot
    \frac{du}{dx}
    ,
    \quad
    \text{onde}
    \quad
    u = y
    \quad
    \text{e}
    \quad
    \frac{d(u^2)}{du}
    =
    2u
    \quad
    \text{:}
\]

\[
    -
    x^2
    \cdot
    \left(
    2y
    \cdot
    \frac{dy}{dx}
    \right)
    -
    \frac{d(x^2)}{dx}
    \cdot
    y^2
    +
    \sin xy
    \cdot
    \left(
    x
    \cdot
    \frac{dy}{dx}
    +
    \frac{dx}{dx}
    \cdot
    y
    \right)
    +
    \frac{dy}{dx}
    =
    \frac{d(4)}{dx}
\]

Simplifiquemos a expressão:

\[
    -
    2
    x^2
    \cdot
    \frac{dy}{dx}
    \cdot
    y
    -
    \frac{d(x^2)}{dx}
    \cdot
    y^2
    +
    \sin xy
    \cdot
    \left(
    x
    \cdot
    \frac{dy}{dx}
    +
    \frac{dx}{dx}
    \cdot
    y
    \right)
    +
    \frac{dy}{dx}
    =
    \frac{d(4)}{dx}
\]

Organizemos a expressão

\[
    -
    \frac{d(x^2)}{dx}
    \cdot
    y^2
    +
    \sin xy
    \cdot
    \left(
    x
    \cdot
    \frac{dy}{dx}
    +
    \frac{dx}{dx}
    \cdot
    y
    \right)
    +
    \frac{dy}{dx}
    -
    2
    x^2
    \cdot
    y
    \cdot
    \frac{dy}{dx}
    =
    \frac{d(4)}{dx}
\]

Usemos as propriedades da função potencial.

\[
    \frac{d(x^n)}{dx}
    =
    nx^{n-1}
    ,
    \quad
    \text{onde}
    \quad
    n = 2
    \quad
    .
    \quad
    \frac{d(x^2)}{dx} = 2x
\]

\[
    -
    y^2
    \cdot
    2x
    +
    \sin xy
    \cdot
    \left(
    x
    \cdot
    \frac{dy}{dx}
    +
    \frac{dx}{dx}
    \cdot
    y
    \right)
    +
    \frac{dy}{dx}
    -
    2
    x^2
    \cdot
    y
    \cdot
    \frac{dy}{dx}
    =
    \frac{d(4)}{dx}
\]

Usemos as propriedades da função potencial.

\[
    \frac{d(x^n)}{dx}
    =
    nx^{n-1}
    ,
    \quad
    \text{onde}
    \quad
    n = 1
    \quad
    .
    \quad
    \frac{d(x)}{dx} = 1
\]

\[
    -
    y^2
    \cdot
    2x
    +
    \sin xy
    \cdot
    \left(
    x
    \cdot
    \frac{dy}{dx}
    +
    1
    \cdot
    y
    \right)
    +
    \frac{dy}{dx}
    -
    2
    x^2
    \cdot
    y
    \cdot
    \frac{dy}{dx}
    =
    \frac{d(4)}{dx}
\]

A derivada da função constante é zero.

\[
    -
    2
    x
    y^2
    +
    \frac{dy}{dx}
    -
    2
    x^2
    \cdot
    y
    \cdot
    \frac{dy}{dx}
    +
    \sin xy
    \cdot
    \left(
    y
    +
    x
    \cdot
    \frac{dy}{dx}
    \right)
    =
    0
\]

Desenvolvemos a expressão

\[
    \sin(xy) \cdot y
    -
    2
    x
    y^2
    +
    \frac{dy}{dx}
    +
    x \cdot \sin(xy) \cdot \frac{dy}{dx}
    -
    2
    x^2
    \cdot
    y
    \cdot
    \frac{dy}{dx}
    =
    0
\]

Subtraímos \(y \cdot \sin(xy) - 2xy^2\) de ambos os lados da igualdade.

\begin{align*}
    y \cdot \sin(xy)
    - y \cdot \sin(xy)
    -
    2
    x
    y^2
    + 2xy^2
    +
    \frac{dy}{dx}
    +
    x \cdot \sin(xy) \cdot \frac{dy}{dx}
    -
    2
    x^2
    \cdot
    y
    \cdot
    \frac{dy}{dx}
     & =
    - y \cdot \sin(xy) + 2xy^2
    \\
    (
    y \cdot \sin(xy)
    - y \cdot \sin(xy)
    )
    +
    (
    -
    2
    x
    y^2
    +
    2xy^2
    )
    +
    \frac{dy}{dx}
    +
    x \cdot \sin(xy) \cdot \frac{dy}{dx}
    -
    2
    x^2
    \cdot
    y
    \cdot
    \frac{dy}{dx}
     & =
    - y \cdot \sin(xy) + 2xy^2
    \\
    (
    0
    )
    +
    (
    0
    )
    +
    \frac{dy}{dx}
    +
    x \cdot \sin(xy) \cdot \frac{dy}{dx}
    -
    2
    x^2
    \cdot
    y
    \cdot
    \frac{dy}{dx}
     & =
    - y \cdot \sin(xy) + 2xy^2
    \\
    \frac{dy}{dx}
    +
    x \cdot \sin(xy) \cdot \frac{dy}{dx}
    -
    2
    x^2
    \cdot
    y
    \cdot
    \frac{dy}{dx}
     & =
    - y \cdot \sin(xy) + 2xy^2
\end{align*}

Fatoramos \(\frac{dy}{dx}\)

\begin{align*}
    \frac{dy}{dx}
    +
    x \cdot \sin(xy) \cdot \frac{dy}{dx}
    -
    2
    x^2
    \cdot
    y
    \cdot
    \frac{dy}{dx}
     & =
    - y \cdot \sin(xy) + 2xy^2
    \\
    \frac{dy}{dx}
    \cdot
    (
    1
    +
    x \cdot \sin(xy)
    -
    2
    x^2
    \cdot
    y
    )
     & =
    - y \cdot \sin(xy) + 2xy^2
\end{align*}

Dividimos ambos os lados por \(
1
+
x \cdot \sin(xy)
-
2
x^2
\cdot
y
\)

\begin{align*}
    \frac{dy}{dx}
    \cdot
    \frac{
        1
        +
        x \cdot \sin(xy)
        -
        2
        x^2
        \cdot
        y
    }{
        1
        +
        x \cdot \sin(xy)
        -
        2
        x^2
        \cdot
        y
    }
     & =
    \frac{
        - y \cdot \sin(xy) + 2xy^2
    }{
        1
        +
        x \cdot \sin(xy)
        -
        2
        x^2
        \cdot
        y
    }
    \\
    \frac{dy}{dx}
    \cdot
    1
     & =
    \frac{
        - y \cdot \sin(xy) + 2xy^2
    }{
        1
        +
        x \cdot \sin(xy)
        -
        2
        x^2
        \cdot
        y
    }
    \\
    \frac{dy}{dx}
     & =
    \frac{
        - y \cdot \sin(xy) + 2xy^2
    }{
        1
        +
        x \cdot \sin(xy)
        -
        2
        x^2
        \cdot
        y
    }
\end{align*}

Com isso, obtemos \(\frac{dy}{dx}\)

\begin{equation}\label{eq:dy_dx}
    \frac{dy}{dx}
    =
    \frac{
        -
        \sin(xy)
        \cdot
        y
        + 2xy^2
    }{
        1
        +
        x \cdot \sin(xy)
        -
        2
        x^2
        \cdot
        y
    }
\end{equation}

Para obter \(\frac{dx}{dy}\), recorreremos à seguinte propriedade,
que vale para as funções com inversa:

\[
    \frac{dx}{dy}
    =
    \frac{1}{\frac{dy}{dx}}
\]

Desse modo

\begin{align*}
    \frac{dx}{dy}
     & =
    \frac{1}{\frac{dy}{dx}}
    \\ &=
    \frac{1}{
        \frac{
            - y \cdot \sin(xy) + 2xy^2
        }{
            1
            +
            x \cdot \sin(xy)
            -
            2
            x^2
            \cdot
            y
        }
    }
    \\ &=
    1
    \cdot
    \frac{
        1
        +
        x \cdot \sin(xy)
        -
        2
        x^2
        \cdot
        y
    }{
        - y \cdot \sin(xy) + 2xy^2
    }
    \\ &=
    \frac{
        1
        +
        x \cdot \sin(xy)
        -
        2
        x^2
        \cdot
        y
    }{
        - y \cdot \sin(xy) + 2xy^2
    }
\end{align*}

Portanto

\begin{equation}
    \frac{dx}{dy}
    =
    \frac{
        1
        +
        x \cdot \sin(xy)
        -
        2
        x^2
        \cdot
        y
    }{
        - y \cdot \sin(xy) + 2xy^2
    }
\end{equation}

\subsubsection*{6. Encontre uma equação da reta tangente e uma equação da reta normal à curva \(y = \sqrt{x} + \frac{1}{x^2}\) no ponto de abscissa \(x=1\).}

Podemos obter uma equação de reta com base em um ponto que faz parte da reta
e o coeficiente angular da reta.

A expressão que permite isso é conhecida como Equação Fundamental
da Reta, e é dada por

\[
    y - y_0 = m (x - x_0)
\]

Em que
\begin{itemize}
    \item \(y_0\) é a ordenada de um ponto conhecido que faz parte da reta
    \item \(x_0\) é a abscissa de um ponto conhecido que faz parte da reta
    \item m é o coeficiente angular da reta
\end{itemize}

Primeiro obteremos um ponto que faz parte da reta.

Com base na abscissa fornecida e na equação da curva,
vamos obter o ponto em que a reta tangente tangencia a curva.
Chamemos esse ponto de P.


\begin{align*}
    P(x_0, y_0) & = P\left(1, \sqrt{1} + \frac{1}{1^2}\right)
    \\
    P(x_0, y_0) & = P\left(1, 1 + 1\right)
    \\
    P(x_0, y_0) & = P(1, 2)
\end{align*}

Portanto, P é o ponto de abscissa \(x_0=1\) e de ordenada \(y_0=2\).

Agora, precisamos de algum mecanismo que nos permita obter
o coeficiente angular da reta tangente à curva.

Nos atentemos ao Significado Geométrico da Diferenciação:

\(f'(x_0)\) representa o coeficiente angular da reta tangente
à curva \(f(x) = y\), no ponto \(P(x_0, f(x_0))\).

De acordo com o Significado Geométrico da Diferenciação,
temos que \(y' = (\sqrt{x} + \frac{1}{x^2})'\) é uma expressão
que fornece o coeficiente angular da reta tangente à curva no
ponto de abcissa \(x\).

Desse modo, devemos obter \(y'\), pois \(y'(1)\) é o coeficiente
angular \(m_t\) da reta tangente que desejamos escrever uma equação.

A derivada da soma é a soma das derivadas.

\begin{equation}\label{eq:q6_soma}
    y'(x) = \frac{d(\sqrt{x})}{dx} + \frac{d(\frac{1}{x^2})}{dx}
\end{equation}

Voltemos nossa atenção para cada termo da soma.

Quanto ao termo \( \frac{d(\sqrt{x})}{dx}\)

\begin{align*}
    \frac{d(\sqrt{x})}{dx}
    =
    \frac{d(x^{\frac{1}{2}})}{dx}
\end{align*}

Exploremos as propriedades da função potencial.

\begin{align*}
    \frac{d(\sqrt{x})}{dx}
    &=
    \frac{1}{2} \cdot x^{\frac{1}{2} - 1}
    \\
    &=
    \frac{1}{2} \cdot x^{-\frac{1}{2}}
    \\
    &=
    \frac{1}{2} \cdot \frac{1}{\sqrt{x}}
    \\
    &=
    \frac{1}{2\sqrt{x}}
\end{align*}

Ou seja

\begin{equation}\label{eq:q6_derivada_1}
    \frac{d(\sqrt{x})}{dx} = \frac{1}{2\sqrt{x}}
\end{equation}

Quanto ao termo \( \frac{d(\frac{1}{x^2})}{dx}\)


\[
    \frac{d(\frac{1}{x^2})}{dx}  = \frac{d(x^{-2})}{dx}
\]

Novamente, exploremos as propriedades da função potencial.

\begin{align*}
    \frac{d(\frac{1}{x^2})}{dx} 
    &= -2x^{-3}
    \\
    &= -2 \cdot \frac{1}{x^3}
    \\
    &= -\frac{2}{x^3}
\end{align*}

Ou seja 

\begin{equation}\label{eq:q6_derivada_2}
    \frac{d(\frac{1}{x^2})}{dx} = -\frac{2}{x^3}
\end{equation}

Das expressões
\ref{eq:q6_soma},
\ref{eq:q6_derivada_1} e
\ref{eq:q6_derivada_2}
decorre que

\begin{equation}\label{eq:q6_derivada_res}
    y'(x) = \frac{1}{2\sqrt{x}} - \frac{2}{x^3}
\end{equation}

Agora podemos calcular \(m_t\)

\[
  m_t 
  =  y'(1) 
  = \frac{1}{2\sqrt{1}} - \frac{2}{1^3} 
  = \frac{1}{2} - 2
  = - \frac{3}{2}
\]

Ou seja

\begin{equation}\label{eq:q6_mt}
    m_t = -\frac{3}{2}
\end{equation}

A reta normal ao gráfico de uma função, em um ponto,
é perpendicular à reta tangente, passando por esse ponto. 
É conhecido que o coeficiente angular da normal é o
oposto do inverso do coeficiente angular da tangente.

Chamemos o coeficiente angular da normal \(m_n\)

\[
m_n 
= - \frac{1}{m_t} 
= - \frac{1}{- \frac{3}{2}}
= - 1 \cdot -\frac{2}{3}
= \frac{2}{3} 
\]

\begin{equation}
    m_n = \frac{2}{3} 
\end{equation}

Observe que ponto P também pertence à reta normal em questão.

Com isso, temos toda a informação necessária para escrever uma equação
da reta tangente e da normal no ponto de abscissa \(x=1\).

Primeiro, obteremos uma equação da reta tangente.

\begin{align*}
    y_t - y_0 
    &= m_t (x_t - x_0)
    \\
    y_t - 2
    &= -\frac{3}{2} (x_t - 1)
    \\
    y_t 
    &= -\frac{3}{2} (x_t - 1) + 2
    \\
    y_t 
    &= -\frac{3}{2}x_t + \frac{3}{2} + 2
    \\
    y_t 
    &= -\frac{3}{2}x_t + \frac{7}{2}
\end{align*}

Depois, obteremos uma equação da reta normal.

\begin{align*}
    y_n - y_0 
    &= m_n (x_n - x_0)
    \\
    y_n - 2
    &= \frac{2}{3} (x_n - 1)
    \\
    y_n
    &= \frac{2}{3} x_n - \frac{2}{3} + 2
    \\
    y_n
    &= \frac{2}{3} x_n + \frac{4}{3}
\end{align*}

Finalmente:

Uma equação da reta tangente à curva \(y = \sqrt{x} + \frac{1}{x^2}\)
no ponto de abscissa \(x=1\) é

\[
    y_t 
    = -\frac{3}{2}x_t + \frac{7}{2}
\]

Uma equação da reta normal à curva \(y = \sqrt{x} + \frac{1}{x^2}\)
no ponto de abscissa \(x=1\) é

\[
    y_n
    = \frac{2}{3} x_n + \frac{4}{3}    
\]


\end{document}
