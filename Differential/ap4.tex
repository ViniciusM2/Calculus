\documentclass{article}
\usepackage[utf8]{inputenc}
\usepackage{cancel}
\usepackage{amsmath}
\usepackage{pgfplots}
\usepackage{amsfonts}
\usepackage{tikz}
\usepackage{float}

\title{AP4 de Cálculo I}
\author{Vinícius Menezes Monte}
\date{Junho de 2022}

\begin{document}

\maketitle

\section*{Questões}

\subsection*{1. Ache as dimensões do maior jardim retangular que pode ser
    fechado com 100 m de cerca.}

\subsection*{2. Suponha que a diminuição na pressão sanguínea de uma pessoa
    dependa de determinada droga que ela deverá tomar. Assim, se \(x \quad mg\) da
    droga forem tomados , a queda da pressão será uma função de \(x\).
    Seja \(f(x)\) essa função e \(f(x) = \frac{1}{2}x^2(k-x)\) onde \(x\) está
    em \([0,k]\), \(k\) constante.
    Calcule o valor de \(x\) que causa o maior decréscimo da pressão.}

\subsection*{3. Analise a função \(f(x) = \frac{1}{4}x^2-2x^3+3x^2+2\), destacando os
    pontos críticos ; os intervalos onde f cresce e onde f decresce; intervalos
    onde a concavidade é positiva e intervalos onde essa concavidade é
    negativa; pontos de inflexão e assíntotas verticais e horizontais se
    essas existirem.}

\section*{Soluções}

\subsection*{1. Ache as dimensões do maior jardim retangular que pode ser
    fechado com 100 m cerca.}

O maior jardim retangular que pode ser fechado com 100 m cerca é
aquele possui a maior área dentre os demais.

A fim de encontrar as dimensões de tal jardim, analisaremos inicialmente
o perímetro do jardim.

Como o jardim é retangular, temos que lados opostos do jardim possuem
o mesmo comprimento. Chamamos o comprimento do primeiro par de lados
opostos \(x\) e o segundo par de lados opostos \(y\), de acordo com a figura:

\begin{figure}[H]
    \centering

    \begin{tikzpicture}

        \node at (-0.5,1.5) {\(x\)};
        \node at (4.5,1.5) {\(x\)};
        \node at (2,-0.5) {\(y\)};
        \node at (2,3.5) {\(y\)};

        \draw (0,0) rectangle (4,3);
    \end{tikzpicture}
    \caption{O jardim é representado por um retângulo de lados \(x\) e \(y\)}
\end{figure}

Como dispomos de 100 metros de cerca, temos que 100 é o perímetro
do retângulo procurado. Desse modo:

\begin{equation}
    2x + 2y = 100
\end{equation}

Ou ainda,

\begin{equation}\label{eq:q1_perimeter}
    x + y = 50
\end{equation}

Agora, vamos analisar a área do jardim.

A área de um retângulo, \(A\), é dada pelo produto do comprimento da base pelo
comprimento da altura. Com os símbolos que escolhemos, temos que:

\begin{equation}\label{eq:q1_area_rectangle_xy}
    A = xy
\end{equation}

A partir de \ref{eq:q1_perimeter}, podemos obter uma expressão que nos dá \(x\) a
partir de \(y\).

\begin{equation}\label{eq:q1_y_from_x}
    y = 50 - x
\end{equation}

Substituindo \ref{eq:q1_y_from_x} em \ref{eq:q1_perimeter}, temos:

\begin{align*}
    A & = xy          \\
    A & = x(50 - x)   \\
    A & = - x^2 + 50x \\
\end{align*}

Agora temos uma função que associa valores de \(x\) a valores de \(A\).

\begin{equation}\label{eq:q1_area_rectangle_y}
    A(x) = - x^2 + 50x
\end{equation}

Perceba que, como não existe área negativa, o domínio da função área
está restrito ao intervalo \([0,50]\). Isso também pode ser visualizado
em um esboço do gráfico dessa função, que é uma parábola com a concavidade para baixo, já que o coeficiente angular é
negativo.

\begin{figure}[H]
    \centering
    \begin{tikzpicture}
        \centering
        \begin{axis}[
                axis x line = center,
                axis y line = center,
                xtick={-10,0,10,...,50},
                % ytick={0,200,...,800},
                ymajorticks=false,
                xlabel style={below right},
                ylabel style={above left},
                xlabel = \(x\),
                ylabel = {\(A(x)\)},
                xmin=-10,
                xmax=60,
                ymin=-200,
                ymax=1000,
            ]
            %Below the red parabola is defined
            \addplot [
                domain=0:50,
                % samples=100,
                color=red,
            ]
            {-x^2 + 50*x};
            \addlegendentry{\(-x^2 + 50x\)}

        \end{axis}
    \end{tikzpicture}
    \caption{Esboço do gráfico da função \(A\)}
\end{figure}


Agora, para estudar os máximos da função, procuremos os números críticos
da função \ref{eq:q1_area_rectangle_y}.

Calculemos \(A'\), a derivada primeira de \(A\).

A derivada da soma é a soma das derivadas.

\[
    A'(x) = (-x^2)' + (50x)'
\]

A Regra do Múltiplo Constante diz que a derivada de uma constante multiplicada
por uma função é a constante multiplicada pela derivada da função.

\[
    A'(x) = -(x^2)' + (50x)'
\]

A derivada da função potencial \(p(x) = x^n\) é  \(p'(x) = nx^{n-1}\). Desse modo:

\begin{align*}
    A'(x)
     & = -(2x^{2-1}) + (50x^{1-1}) \\
     & = -(2x^{1}) + (50x^{0})     \\
     & = -(2x^{1}) + (50x^{0})     \\
     & = -(2x) + (50\cdot 1)       \\
     & = -2x + 50                  \\
\end{align*}

Desse modo, temos \(A'\)

\begin{equation}\label{eq:q1_derivative_of_area}
    A'(x) = -2x + 50
\end{equation}

Agora, procuremos os valores de \(x\) que fazem \(A'(x) = 0\). Esses valores
serão números críticos, candidatos a gerarem máximos e mínimos.


\begin{align*}
    A'(x)
             & = 0  \\
    -2x + 50 & = 0  \\
    2x       & = 50 \\
    x        & = 25 \\
\end{align*}

Ou seja, temos que \(25\) é um número crítico.

Para encontrar os máximos e mínimos de \(A\) testamos os números críticos
e os extremos do intervalo.

% A(x) = - x^2 + 50x
\begin{itemize}
    \item \(A(0) = - (0)^2 + 50(0) = 0 + 0 = 0 \)
    \item \(A(25) = -25^2 + 50(25) = -625 + 1250 = 625  \)
    \item \(A(50) = - (50)^2 + 50(50) = -2500 + 2500 = 0  \)
\end{itemize}

Ou seja, \(x = 25\) gera área máxima.

\begin{equation}\label{eq:q1_x_max}
    x_{max} = 25
\end{equation}

Através das expressões \ref{eq:q1_y_from_x} e \ref{eq:q1_x_max},
obtemos \(y_{max}\).

\[
    y_{max} = 50 - x_{max} = 50 - 25 = 25
\]

Ou seja, o valor de \(y\) para a maior área é \(25\).

\begin{equation}\label{eq:q1_y_max}
    y_{max} = 25
\end{equation}

Finalmente, as dimensões do maior jardim retangular que pode ser
fechado com 100 metros de cerca são \(25 \times 25\). Ou seja, tal jardim é quadrado.

\begin{figure}[H]
    \centering

    \begin{tikzpicture}

        \node at (-0.5,2) {\(25 \ m\)};
        \node at (4.5,2) {\(25 \ m\)};
        \node at (2,-0.5) {\(25 \ m\)};
        \node at (2,4.5) {\(25 \ m\)};

        \draw (0,0) rectangle (4,4);
    \end{tikzpicture}
    \caption{O jardim é quadrado}
\end{figure}

\subsection*{
    2. Suponha que a diminuição na pressão sanguínea de uma pessoa
    dependa de determinada droga que ela deverá tomar. Assim, se \(x \quad mg\) da
    droga forem tomados , a queda da pressão será uma função de \(x\).
    Seja \(f(x)\) essa função e \(f(x) = \frac{1}{2}x^2(k-x)\) onde \(x\) está
    em \([0,k]\), \(k\) constante.
    Calcule o valor de \(x\) que causa o maior decréscimo da pressão.
}

Inicialmente, vamos trabalhar a expressão de \(f\) de modo a achar \(g = f\)
que torne o estudo mais simples.

\begin{align*}
    \frac{1}{2}x^2(k-x) = \frac{k}{2}x^2 - \frac{1}{2}x^3
\end{align*}

Desse modo:

\begin{equation}
    g(x) = \frac{k}{2}x^2 - \frac{1}{2}x^3
\end{equation}

Agora, obtenhamos \(g'\), a derivada primeira de \(g\).

A derivada da soma é a soma das derivadas.

\[
    g'(x) = \left(\frac{k}{2}x^2\right)' + \left(-\frac{1}{2}x^3\right)'
\]

A Regra do Múltiplo Constante diz que a derivada de uma constante multiplicada
por uma função é a constante multiplicada pela derivada da função.

\[
    g'(x) = \frac{k}{2}(x^2)' -\frac{1}{2}(x^3)'
\]

A derivada da função potencial \(p(x) = x^n\) é  \(p'(x) = nx^{n-1}\). Desse modo:

\begin{align*}
    g'(x)
     & = \frac{k}{2}(x^2)' -\frac{1}{2}(x^3)'                  \\
     & = \frac{k}{2}(2x^{2-1}) -\frac{1}{2}(3x^{3-1})          \\
     & = \frac{k}{2}(2x^{1}) -\frac{1}{2}(3x^{2})              \\
     & = \frac{k}{\cancel{2}}(\cancel{2}x) -\frac{3}{2}(x^{2}) \\
     & = kx -\frac{3}{2}x^{2}                                  \\
\end{align*}

Ou seja,

\begin{equation}\label{eq:q2_derivative_of_pressure}
    g'(x) = kx -\frac{3}{2}x^{2}
\end{equation}

Agora, procuremos os valores de \(x\) que fazem \(g'(x) = 0\). Esses valores
serão números críticos, candidatos a gerarem máximos e mínimos.

\begin{align*}
    g'(x)
                         & = 0 \\
    kx -\frac{3}{2}x^{2} & = 0 \\
\end{align*}

Percebemos que a expressão dos valores de \(x\) que fazem \(g'(x) = 0\) é
uma equação do segundo grau do tipo \(ax^2 + bx = 0\). Podemos obter as raízes
de tal equação através do processo de fatoração:

\begin{align*}
    kx -\frac{3}{2}x^{2}         & = 0 \\
    x\left(k-\frac{3}{2}x\right) & = 0 \\
\end{align*}

É imediato que a equação possui uma raiz nula \(x_1 = 0\). Podemos obter a segunda
raiz através da solução da seguinte equação:

\[
    k-\frac{3}{2}x_2 = 0
\]

Resolvemos a equação, de acordo:

\begin{align*}
    k-\frac{3}{2}x_2 & = 0            \\
    -\frac{3}{2}x_2  & = -k           \\
    \frac{3}{2}x_2   & = k            \\
    x_2              & = \frac{2}{3}k
\end{align*}

Com isso, obtemos a segunda raiz \(x_2 = \frac{2}{3}k\).

Para encontrar os máximos e mínimos de \(f = g\) testamos os números críticos
e os extremos do intervalo.

\begin{itemize}
    \item Início do intervalo \(x = 0\)
          \[f(0) = \frac{1}{2}0^2(k-0) = 0k = 0\]
    \item Crítico: \(x = 0\)
          \[f(0) = \frac{1}{2}0^2(k-0) = 0k = 0\]
    \item Crítico: \(x = \frac{2}{3}k\)
          \begin{align*}
              f
              \left(
              \frac{2}{3}k
              \right)
               & =
              \frac{1}{2}
              \left(
              \frac{2}{3}k
              \right)
              ^2
              \left[
                  k-
                  \left(
                  \frac{2}{3}k
                  \right)
                  \right]
              \\
               & =
              \frac{1}{2}
              \left(
              \frac{4}{9}k^2
              \right)
              \left(
              k-
              \frac{2}{3}k
              \right)
              \\
               & =
              \frac{4}{18}
              k^2
              \left(
              k-
              \frac{2}{3}k
              \right)
              \\
               & =
              \frac{2}{9}
              k^2
              \left(
              k-
              \frac{2}{3}k
              \right)
              \\
               & =
              \frac{2}{9}
              k^3
              -
              \frac{4}{27}k^3
              \\
               & =
              \frac{6}{27}
              k^3
              -
              \frac{4}{27}k^3
              \\
               & =
              \frac{2}{27}
              k^3
              \\
          \end{align*}
    \item Fim do intervalo \(x = k\)
          \[f(k) = \frac{1}{2}k^2(k-k) = \frac{1}{2}k^2(0) = 0 \]
\end{itemize}

Segue que o valor de \(x\) que causa o maior decréscimo de pressão é:

\[
    x_{max} = \frac{2}{27} k^3
\]

\subsection*{3. Analise a função \(f(x) = \frac{1}{4}x^4-2x^3+3x^2+2\), destacando os
    pontos críticos ; os intervalos onde f cresce e onde f decresce; intervalos
    onde a concavidade é positiva e intervalos onde essa concavidade é
    negativa; pontos de inflexão e assíntotas verticais e horizontais se
    essas existirem.}


Começamos derivando \(f\).

A derivada da soma é a soma das derivadas.

\[
    f'(x) = \left(\frac{1}{4}x^4\right)'+(-2x^3)'+(3x^2)'+(2)'
\]

A Regra do Múltiplo Constante diz que a derivada de uma constante multiplicada
por uma função é a constante multiplicada pela derivada da função.

\[
    f'(x) = \frac{1}{4}(x^4)'-2(x^3)'+ 3(x^2)' + (2)'
\]

A derivada da constante é zero.

\[
    f'(x) = \frac{1}{4}(x^4)'-2(x^3)'+ 3(x^2)' + 0 = \frac{1}{4}(x^4)'-2(x^3)'+ 3(x^2)'
\]

A derivada da função potencial \(p(x) = x^n\) é  \(p'(x) = nx^{n-1}\). Desse modo:

\begin{align*}
    f'(x)
     & = \frac{1}{4}(x^4)'-2(x^3)'+ 3(x^2)'                   \\
     & = \frac{1}{4}(4x^{4-1})-2(3x^{3-1})+ 3(2x^{2-1})       \\
     & = \frac{1}{4}(4x^{3})-2(3x^2)+ 3(2x^1)                 \\
     & = \frac{1}{\cancel{4}}(\cancel{4}x^{3})-2(3x^2)+ 3(2x) \\
     & = x^{3}-6x^2+ 6x                                       \\
\end{align*}

Portanto:

\begin{equation}\label{eq:q3_derivative_f}
    f'(x) = x^{3}-6x^2+ 6x
\end{equation}

Temos interesse nos pontos críticos de \(f\). Os pontos críticos de \(f\) são
valores no domínio de \(f\) para os quais \(f\) não é diferenciável ou sua
derivada é \(0\).

Uma vez que \(f\) é polinomial, temos que \(f\) é derivável em todos os seus
pontos. Desse modo, os pontos críticos de \(f\) são apenas os valores no
domínio de \(f\) para os quais a sua derivada é \(0\).

\begin{align*}
    f'(x)          & = 0 \\
    x^{3}-6x^2+ 6x & = 0 \\
\end{align*}

Ou seja, os pontos críticos de \(f\) são as raízes da equação:

\[
    x^{3}-6x^2+ 6x = 0
\]

Começemos o processo de resolução dessa equação, então.

\begin{align*}
    x^{3}-6x^2+ 6x & = 0 \\
    x(x^2-6x+6)    & = 0 \\
\end{align*}

É imediato que a equação possui uma raiz nula \(x_1 = 0\). Podemos obter a segunda
raiz através da solução da seguinte equação:

\[
    x^2-6x+6 = 0
\]

Resolvemos essa equação, então. Primeiro, adicionamos \(3\) em ambos os lados.

\begin{align*}
    x^2-6x+9 & = 3 \\
\end{align*}

Escrevemos o lado esquerdo como um quadrado.

\begin{align*}
    (x-3)^2 & = 3 \\
\end{align*}

Aplicamos a raiz quadrada em ambos os lados.

\begin{align*}
    \sqrt{(x-3)^2} & = \sqrt{3} \\
\end{align*}

Por definição, a raiz quadrada do quadrado de x é o módulo de x.

\begin{align*}
    |x-3| & = \sqrt{3} \\
\end{align*}

Vamos nos recordar a definição de \(|x|\).

\begin{equation} \label{modulus_definition}
    |x| =
    \left\{
    \begin{array}{ll}
        x  & \mbox{se } x \geq 0 \\
        -x & \mbox{se } x < 0
    \end{array}
    \right.
\end{equation}

Desse modo:

\begin{equation}
    |x - 3| =
    \left\{
    \begin{array}{ll}
        x - 3    & \mbox{se } x - 3 \geq 0 \\
        -(x - 3) & \mbox{se } x - 3 < 0
    \end{array}
    \right.
\end{equation}

Vamos obter, então, a raiz correspondente a cada caso:

\begin{itemize}
    \item \(|x - 3| = x - 3\)
          \begin{align*}
              |x-3| & = \sqrt{3}     \\
              x - 3 & = \sqrt{3}     \\
              x     & = 3 + \sqrt{3} \\
          \end{align*}
    \item \(|x - 3| = -(x - 3)\)
          \begin{align*}
              |x-3|    & = \sqrt{3}    \\
              -(x - 3) & = \sqrt{3}    \\
              -x + 3   & = \sqrt{3}    \\
              x - 3    & = -\sqrt{3}   \\
              x        & = 3 -\sqrt{3} \\
          \end{align*}
\end{itemize}

Com isso, obtivemos os pontos críticos de \(f\). Eles são:
\begin{itemize}
    \item \(x_0' = 0\)
    \item \(x_0'' = 3 + \sqrt{3}\)
    \item \(x_0''' = 3 - \sqrt{3}\)
\end{itemize}

Agora, vamos analisar os intervalos nos quais \(f\) cresce e decresce.

O Teorema do Fator diz que se \(x = c\) é uma raiz de um polinômio, então
\((x - c)\) é um fator desse polinômio.

Podemos aplicar isso à expressão de \(f'\) a fim de obter uma fatoração que
nos permita estudar o final de \(f'\).

\begin{itemize}
    \item Se 0 é raiz do polinômio, então \(x-0 = x\) é fator do polinômio
    \item Se \( 3 + \sqrt{3}\) é raiz do polinômio,
          então \(x-(3 + \sqrt{3}) = x - 3 - \sqrt{3}\) é fator do polinômio
    \item Se \( 3 - \sqrt{3}\) é raiz do polinômio,
          então \(x-(3 - \sqrt{3}) = x - 3 + \sqrt{3}\) é fator do polinômio
\end{itemize}

Desse modo, temos que:

\begin{equation}
    x^{3}-6x^2+ 6x = x(x - 3 - \sqrt{3})(x - 3 + \sqrt{3})
\end{equation}

Agora, vamos desenhar um esquema para estudar o comportamento da função:

% \sqrt{3} \approx 1.73
% \sqrt{3} + 3 \approx 4.73
% -\sqrt{3} + 3 \approx 1,27

\begin{figure}[H]
    \centering
    \begin{tikzpicture}

        % vertical lines
        \draw[orange,thick,dashed] (3,0) -- (3,-6) ;
        \draw[orange,thick,dashed] (4.27,0) -- (4.27,-6) ;
        \draw[orange,thick,dashed] (7.73,0) -- (7.73,-6) ;


        %%%%%%%%%%%%%%%%%%%%%5

        % function
        \node (x) at (-1,0) {\(x\)};
        % point
        \filldraw [black] (3,0) circle (2pt);
        % value of point
        \node (0) at (3,0.35) {\(0\)};
        % reta
        \draw [-to](0,0) -- (10,0);
        % signals
        \node (x) at (1.5,1.0) {\(\textcolor{blue}{-}\)};
        \node (x) at (3.65,1.0) {\(\textcolor{red}{+}\)};
        \node (x) at (6,1.0) {\(\textcolor{red}{+}\)};
        \node (x) at (9,1.0) {\(\textcolor{red}{+}\)};
        %%%%%%%%%%%%%%%%%%%%
        % \sqrt{3} \approx 1.73
        % \sqrt{3} + 3 \approx 4.73
        % -\sqrt{3} + 3 \approx 1,27
        % point
        \filldraw [black] (7.73,-2) circle (2pt);
        % value of point
        \node (0) at (7.73,-1.65) {\((\sqrt{3} + 3)\)};
        % function
        \node (x) at (-1,-2) {\(x - 3 - \sqrt{3}\)};
        % reta
        \draw [-to](0,-2) -- (10,-2);
        % signals
        \node (x) at (1.5,-1.0) {\(\textcolor{blue}{-}\)};
        \node (x) at (3.65,-1.0) {\(\textcolor{blue}{-}\)};
        \node (x) at (6,-1.0) {\(\textcolor{blue}{-}\)};
        \node (x) at (9,-1.0) {\(\textcolor{red}{+}\)};
        %%%%%%%%%%%%%%%%%%%%
        % point
        \filldraw [black] (4.27,-4) circle (2pt);
        % value of point
        \node (0) at (4.27,-3.65) {\((-\sqrt{3} + 3)\)};
        % function
        \node (x) at (-1,-4) {\(x - 3 + \sqrt{3}\)};
        % reta
        \draw [-to](0,-4) -- (10,-4);
        % signals
        \node (x) at (1.5,-3.0) {\(\textcolor{blue}{-}\)};
        \node (x) at (3.65,-3.0) {\(\textcolor{blue}{-}\)};
        \node (x) at (6,-3.0) {\(\textcolor{red}{+}\)};
        \node (x) at (9,-3.0) {\(\textcolor{red}{+}\)};
        %%%%%%%%%%%%%%%%%%%%
        % function
        \node (x) at (-1,-6) {\(f'(x)\)};
        % point
        \filldraw [black] (3,-6) circle (2pt);
        % value of point
        \node (0) at (3,-5.65) {\(0\)};
        % point
        \filldraw [black] (7.73,-6) circle (2pt);
        % value of point
        \node (0) at (7.73,-5.65) {\((\sqrt{3} + 3)\)};
        % point
        \filldraw [black] (4.27,-6) circle (2pt);
        % value of point
        \node (0) at (4.27,-5.65) {\((-\sqrt{3} + 3)\)};
        % reta
        \draw [-to](0,-6) -- (10,-6);
        % signals
        % signals
        \node (x) at (1.5,-5.0) {\(\textcolor{blue}{-}\)};
        \node (x) at (3.65,-5.0) {\(\textcolor{red}{+}\)};
        \node (x) at (6,-5.0) {\(\textcolor{blue}{-}\)};
        \node (x) at (9,-5.0) {\(\textcolor{red}{+}\)};    \end{tikzpicture}
    \caption{Intervalos nos quais \(f\) cresce e decresce.}
    \label{fig:q3_intervals_f}
\end{figure}

Como se pode ver, a função cresce no intervalo
\(  (0, -\sqrt{3} + 3) \cup (\sqrt{3} + 3, +\infty)  \)
e descresce no intervalo
\(  (-\infty, 0) \cup (-\sqrt{3} + 3, \sqrt{3} + 3)   \).



Agora, investiguemos os intervalos onde a concavidade é positiva e
onde a concavidade é negativa.
A concavidade da função \(f\) é positiva quando \(f''(x)>0\)
e negativa quando \(f''(x)<0\).

Que obtenhamos \(f''\).

\begin{align*}
    f''(x)
     & = (f(x)')'          \\
     & = (x^{3}-6x^2+ 6x)' \\
     & = 3x^{2}-12x+ 6     \\
\end{align*}

Portanto,

\begin{equation}
    f''(x) = 3x^{2}-12x+ 6
\end{equation}

Vamos estudar o sinal de \(f''\) para descobrir o sinal da
concavidade em cada intervalo.

Primeiro, obtemos as raízes de \(f''\), resolvendo a seguinte equação:

\[
    3x^{2}-12x+ 6 = 0
\]

Dividimos ambos os lados por \(3\).

\[
    x^{2}-4x+ 2 = 0
\]

Adicionamos \(2\) em ambos os lados da equação.

\[
    x^{2}-4x+ 4 = 2
\]

Escrevemos o lado esquerdo da equação como um quadrado.

\[
    (x-2)^2 = 2
\]

Aplicamos raiz quadrada em ambos os lados da equação.

\[
    \sqrt{(x-2)^2} = \sqrt{2}
\]

Por definição, temos que:

\[
    |x-2| = \sqrt{2}
\]

Os valores do lado esquerdo da equação seguem a definição de módulo,
divididos em casos.

\begin{equation}
    |x - 2| =
    \left\{
    \begin{array}{ll}
        x - 2    & \mbox{se } x - 2 \geq 0 \\
        -(x - 2) & \mbox{se } x - 2 < 0
    \end{array}
    \right.
\end{equation}

Vamos obter, então, a raiz correspondente a cada caso:

\begin{itemize}
    \item \(|x - 2| = x - 2\)
          \begin{align*}
              |x - 2| & = \sqrt{2}     \\
              x - 2   & = \sqrt{2}     \\
              x       & = 2 + \sqrt{2} \\
          \end{align*}
    \item \(|x - 2| = -(x - 2)\)
          \begin{align*}
              |x - 2|  & = \sqrt{2}    \\
              -(x - 2) & = \sqrt{2}    \\
              -x + 2   & = \sqrt{2}    \\
              x - 2    & = -\sqrt{2}   \\
              x        & = 2 -\sqrt{2} \\
          \end{align*}
\end{itemize}

Ou seja, as raízes de \(3x^{2}-12x+ 6\) são \(x_1 = 2 -\sqrt{2}\) e \(x_2 = 2 + \sqrt{2}\).

Utilizemos o Teorema do Fator para fatorar o polinômio.

\begin{itemize}
    \item Se \( 2 + \sqrt{2}\) é raiz do polinômio,
          então \(x-(2 + \sqrt{2}) = x - 2 - \sqrt{2}\) é fator do polinômio
    \item Se \( 2 - \sqrt{2}\) é raiz do polinômio,
          então \(x-(2 - \sqrt{2}) = x - 2 + \sqrt{2}\) é fator do polinômio
\end{itemize}

Desse modo, temos que:

\begin{equation}
    3x^{2}-12x+ 6 = 3(x - 2 - \sqrt{2})(x - 2 + \sqrt{2})
\end{equation}

Agora, vamos desenhar um esquema para estudar o comportamento da função:

\begin{figure}[H]
    \centering
    \begin{tikzpicture}

        % vertical lines
        \draw[orange,thick,dashed] (3,0) -- (3,-6) ;
        \draw[orange,thick,dashed] (4.27,0) -- (4.27,-6) ;
        \draw[orange,thick,dashed] (7.73,0) -- (7.73,-6) ;

        % point
        \filldraw [black] (7.73,-2) circle (2pt);
        % value of point
        \node (0) at (7.73,-1.65) {\((\sqrt{2} + 2)\)};
        % function
        \node (x) at (-1,-2) {\(x - 2 - \sqrt{2}\)};
        % reta
        \draw [-to](0,-2) -- (10,-2);
        % signals
        \node (x) at (1.5,-1.0) {\(\textcolor{blue}{-}\)};
        \node (x) at (3.65,-1.0) {\(\textcolor{blue}{-}\)};
        \node (x) at (6,-1.0) {\(\textcolor{blue}{-}\)};
        \node (x) at (9,-1.0) {\(\textcolor{red}{+}\)};
        %%%%%%%%%%%%%%%%%%%%
        % point
        \filldraw [black] (4.27,-4) circle (2pt);
        % value of point
        \node (0) at (4.27,-3.65) {\((-\sqrt{2} + 2)\)};
        % function
        \node (x) at (-1,-4) {\(x - 2 + \sqrt{2}\)};
        % reta
        \draw [-to](0,-4) -- (10,-4);
        % signals
        \node (x) at (1.5,-3.0) {\(\textcolor{blue}{-}\)};
        \node (x) at (3.65,-3.0) {\(\textcolor{blue}{-}\)};
        \node (x) at (6,-3.0) {\(\textcolor{red}{+}\)};
        \node (x) at (9,-3.0) {\(\textcolor{red}{+}\)};
        %%%%%%%%%%%%%%%%%%%%
        % function
        \node (x) at (-1,-6) {\(f'(x)\)};
        % point
        \filldraw [black] (3,-6) circle (2pt);
        % value of point
        \node (0) at (3,-5.65) {\(0\)};
        % point
        \filldraw [black] (7.73,-6) circle (2pt);
        % value of point
        \node (0) at (7.73,-5.65) {\((\sqrt{2} + 2)\)};
        % point
        \filldraw [black] (4.27,-6) circle (2pt);
        % value of point
        \node (0) at (4.27,-5.65) {\((-\sqrt{2} + 2)\)};
        % reta
        \draw [-to](0,-6) -- (10,-6);
        % signals
        % signals
        \node (x) at (1.5,-5.0) {\(\textcolor{red}{+}\)};
        \node (x) at (3.65,-5.0) {\(\textcolor{red}{+}\)};
        \node (x) at (6,-5.0) {\(\textcolor{blue}{-}\)};
        \node (x) at (9,-5.0) {\(\textcolor{red}{+}\)};    \end{tikzpicture}
    \caption{Intervalos nos quais \(f\) cresce e decresce.}
    \label{fig:q3_intervals_f_part_2}
\end{figure}

Como se pode ver, a concavidade é positiva no intervalo
\(  (-\infty, -\sqrt{2} + 2) \cup (\sqrt{2} + 2, +\infty)  \)
e negativa no intervalo
\(  (-\sqrt{2} + 2, \sqrt{2} + 2)   \).

Na posse do esquema \ref{fig:q3_intervals_f_part_2}, também é possível obter
facilmente os pontos de inflexão.
Pontos de Inflexão de \(f\) são os pontos em \(f\) em que a curvatura(a derivada de segunda ordem)
muda de sinal.
Portanto, os pontos de inflexão de \(f\) são:
\begin{itemize}
    \item \(A(-\sqrt{2} + 2, f(-\sqrt{2} + 2)) = A(-\sqrt{2} + 2, 4\sqrt{2} - 3)\)
    \item \(B(\sqrt{2} + 2, f(\sqrt{2} + 2)) = B(\sqrt{2} + 2, -3-4\sqrt{2}) \)
\end{itemize}

Agora, verifiquemos se a função admite assíntotas.

O gráfico da função \(f\) admite assíntota vertical se

\[
    \lim_{x\to x_0} = \pm \infty
    \qquad \mbox{e} \qquad
    f(x_0) = \frac{k}{0} \mbox{, } k \neq 0
\]

Além disso, nessas condições, a assíntota vertical é \(x = x_0\).

É imediato que \(f\) \textbf{não} admite assíntota vertical, pois \(f\) é
polinomial, seu domínio é \(\mathbb{R}\), e ela é contínua em todos
os pontos, jamais ocorrendo saltos ou buracos.
Outro argumento é que a expressão não possui denominador que pudesse ser zerado, tal que a condição
\(f(x_0) = \frac{k}{0} \mbox{, } k \neq 0\) fosse satisfeita.

Por outro lado, o gráfico da função \(f\) admite assíntota horizontal se
\[
    \lim_{x\to \pm \infty} f = L
\]

em que \(L\) é uma constante real.

Verifiquemos se \(f\) admite assíntota horizontal.
Para isso, calculamos o limite mencionado.

\begin{align*}
    \lim_{x\to \pm \infty} \frac{1}{4}x^4-2x^3+3x^2+2
    &=
    \lim_{x\to \pm \infty} \frac{1}{4}x^4 \\
    &=
    \lim_{x\to \pm \infty} x^4 \\
\end{align*}

Como quadrados são sempre números positivos, segue que 

\begin{equation}
    \lim_{x\to \pm \infty} \frac{1}{4}x^4-2x^3+3x^2+2 = +\infty
\end{equation}

Portanto, \(f\) \textbf{não} admite assíntota horizontal.

Finalmente, podemos resumir as informações desejadas.


Pontos Críticos:
\begin{itemize}
    \item \(x_0' = 0\)
    \item \(x_0'' = 3 + \sqrt{3}\)
    \item \(x_0''' = 3 - \sqrt{3}\)
\end{itemize}

Intervalos onde \(f\) cresce e decresce:
\begin{itemize}
    \item cresce no intervalo \(  (0, -\sqrt{3} + 3) \cup (\sqrt{3} + 3, +\infty)  \)
    \item descresce no intervalo \(  (-\infty, 0) \cup (-\sqrt{3} + 3, \sqrt{3} + 3)   \)
\end{itemize}

Intervalos onde a concavidade é positiva ou é negativa:
\begin{itemize}
    \item positiva no intervalo \(  (-\infty, -\sqrt{2} + 2) \cup (\sqrt{2} + 2, +\infty)  \)
    \item negativa no intervalo \(  (-\sqrt{2} + 2, \sqrt{2} + 2)   \)
\end{itemize}

Pontos de Inflexão:
\begin{itemize}
    \item \( A(-\sqrt{2} + 2, 4\sqrt{2} - 3)\)
    \item \( B(\sqrt{2} + 2, -3-4\sqrt{2}) \)
\end{itemize}

Sobre assíntotas verticais e horizontais:
\begin{itemize}
    \item \(f\) \textbf{não} admite assíntota vertical
    \item \(f\) \textbf{não} admite assíntota horizontal
\end{itemize}
% \subsection*{3. Analise a função \(f(x) = \frac{1}{4}x^4-2x^3+3x^2+2\), destacando os
%     pontos críticos ; os intervalos onde f cresce e onde f decresce; intervalos
%     onde a concavidade é positiva e intervalos onde essa concavidade é
%     negativa; pontos de inflexão e assíntotas verticais e horizontais se
%     essas existirem.}

\end{document}
